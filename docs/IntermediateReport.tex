\documentclass[times,9pt,article]{llncs}
\usepackage{times}
\usepackage{makeidx}

\begin{document}
\title{Peer-to-Peer File System}
\author{Group Number 2 \\
Jo\~ao Granchinho n.54766 mail@ist.utl.pt \\
Pedro Torres  n.63506 mail@ist.utl.pt \\
Rodrigo Bruno n.67074 rodrigo.bruno@ist.utl.pt}
\institute{Peer-to-Peer Systems and Overlay Networks \\
Master Degree in Telecommunications and Informatics Engineering \\
Superior Technical Institute}
\maketitle


%%%%%%%%%%%%%%%%%%%%%%%%%%%%%%%%%%%%%%%%%%%%%%%%%%%%%%%%%%%%%%%%%%%%%%%%%%%%%%%
% Section 1:  Introduction 
%%%%%%%%%%%%%%%%%%%%%%%%%%%%%%%%%%%%%%%%%%%%%%%%%%%%%%%%%%%%%%%%%%%%%%%%%%%%%%%
\section{Introduction}
In this document we will be describing our solution and explaining how we intend
to fulfill the challenged proposed within the project specification.

We will start by explaining why we choose Kademlia as our DHT for this project.
Next we detail the protocol that we will be using for implementing a distributed
file system on top of a Kademlia peer-to-peer network overlay.

Conclusion?

%%%%%%%%%%%%%%%%%%%%%%%%%%%%%%%%%%%%%%%%%%%%%%%%%%%%%%%%%%%%%%%%%%%%%%%%%%%%%%%
% Section 2:  Kademlia 
%%%%%%%%%%%%%%%%%%%%%%%%%%%%%%%%%%%%%%%%%%%%%%%%%%%%%%%%%%%%%%%%%%%%%%%%%%%%%%%
\section{Kademlia}

As one of the studied DHT and one of the proposed DHT implementations for the 
project, we decided to use TomP2P, a Kademlia's protocol implementation.

This decision was mainly motivated by the fact that Kademlia was designed to be
used by file sharing applications and therefore provides some nice features that
will be very helpful for our file system's implementation. 

We will now describe some of the Kademlia features and explain how we will take 
advantage of them. 

\subsection{Iterative Parallel Search}
Kademlia and therefore TomP2P uses iterative parallel search. Two main benefits
from this search procedure are: 

\begin{itemize}
\item generated/received information is useful for refreshing the k-buckets;
\item parallel queries prevents waiting for timeouts to detect failed nodes and 
allows the fastest nodes (the ones with the lowest RTT) to be used. 
\end{itemize} 

As that being, using Kademlia we will be able to provide a better quality of
service by providing faster search and reduced maintenance traffic.

\subsection{Key-Republishing}
Key-Republishing is a very interesting feature provided by Kademlia. This feature
is important to ensure the persistence of the key-value pairs.

Two phenomena may jeopardize the key-value pairs: a node responsible for the pair
leaving the network and a node with a closer id (closer to the key) joining the
network. 

TomP2P takes care of both scenarios using Indirect Replication (name used in the
TomP2P's documentation). The activation of this mechanism will ensure that nodes
react when one of the situations above described happen. 

This Key-Republishing will be very useful since it will help implementing file
replication algorithm that is one of the requirements for our file system.   

\subsection{Caching (direct replication)}

Another interesting feature is caching. Caching enables the replication of 
key-value pairs along the search path. If we take in consideration that our 
distance metric (exclusive or) is unidirectional, this means that searches 
from different origins will converge into a search path. As that being, using
caching, Kademlia is able to reduce the burden on nodes holding popular content.

Within the context of our project, this functionality can be used to speed up
the statistical information delivery. Since this information will be calculated
periodically, its contents can be cached (temporarily) to provide lower load on
the node holding this information. 

\end{document}
